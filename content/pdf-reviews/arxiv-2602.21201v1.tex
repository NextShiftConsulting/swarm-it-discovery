\documentclass[11pt,a4paper]{article}
\usepackage[utf8]{inputenc}
\usepackage[margin=1in]{geometry}
\usepackage{hyperref}
\usepackage{graphicx}
\usepackage{amsmath}
\usepackage{booktabs}

\title{RSCT Review: Aletheia tackles FirstProof autonomously}
\author{Swarm-It Research Discovery\\\small Automated RSCT-Based Analysis}
\date{February 25, 2026}

\begin{document}
\maketitle

\begin{abstract}
This document provides an automated review of the paper ``Aletheia tackles FirstProof autonomously'' from an RSCT (Representation-Solver Compatibility Testing) perspective. The paper achieved an RSCT relevance score of 31.7% and topic similarity of 54.4%.
\end{abstract}

\section{Paper Information}
\begin{itemize}
\item \textbf{Title:} Aletheia tackles FirstProof autonomously
\item \textbf{Authors:} Tony Feng, Junehyuk Jung, Sang-hyun Kim, Carlo Pagano, Sergei Gukov et al.
\item \textbf{Source:} \url{https://arxiv.org/abs/2602.21201v1}
\item \textbf{RSCT Relevance:} 31.7%
\item \textbf{Topic Match:} 54.4%
\item \textbf{Key Concepts:} N/A
\end{itemize}

\section{Original Abstract}
We report the performance of Aletheia (Feng et al., 2026b), a mathematics research agent powered by Gemini 3 Deep Think, on the inaugural FirstProof challenge. Within the allowed timeframe of the challenge, Aletheia autonomously solved 6 problems (2, 5, 7, 8, 9, 10) out of 10 according to majority expert assessments; we note that experts were not unanimous on Problem 8 (only). For full transparency, we explain our interpretation of FirstProof and disclose details about our experiments as well as our evaluation. Raw prompts and outputs are available at https://github.com/google-deepmind/superhuman/tree/main/aletheia.

\section{RSCT Analysis}
**Summary**

The paper titled "Aletheia tackles FirstProof autonomously" reports on the performance of the Aletheia mathematics research agent, driven by the advanced capabilities of the Gemini 3 Deep Think framework, as it participated in the FirstProof challenge, where it autonomously solved six out of ten mathematical problems. The authors provide a detailed account of Aletheia's performance metrics, noting disparities in expert evaluations, particularly regarding Problem 8, which had mixed assessments. This nuance in expert critiques is significant, as it highlights the subjective nature of evaluating automated systems in complex problem-solving contexts. Furthermore, the authors promise full transparency by allowing access to their raw prompts and outputs via a publicly available GitHub repository, contributing to open science and reproducibility in AI research.

**RSCT Analysis**

From a Representation-Solver Compatibility Testing (RSCT) perspective, this work offers significant insights but lacks comprehensive engagement with core RSCT principles. Firstly, the paper does not explicitly address the quality of the representations used by Aletheia (R). While it showcases the agent's problem-solving capabilities, a critical evaluation of whether the underlying representations accurately captured the mathematical concepts inherent in the FirstProof challenge seems to be overlooked. Secondly, the paper does not appear to handle spurious correlations (S), which are vital for ensuring the robustness of AI systems in making decisions based on learned data. The simplistically framed expert evaluations do not provide a detailed analysis regarding the presence of such correlations that may lead to misinterpretations of Aletheia's true capabilities. 

On the issue of noise and uncertainty (N), the study fails to address the potential impact of variability in expert assessments on Aletheia’s evaluation. Given that Problem 8 elicited divided opinions, an in-depth analysis of how noise in expert perceptions might affect the overall assessment of Aletheia’s performance is absent. Lastly, concerning the kappa compatibility metric, which quantitatively measures the degree of agreement among expert evaluators, it would have been beneficial to include a discussion on how such inter-rater reliability might influence the perceived effectiveness of Aletheia. The lack of these considerations limits the RSCT's applicability to this discourse, ultimately showing gaps in how the work could offer significant understanding of agent representation and problem-solving.

**Technical Depth**

The methodological contributions of the paper lie in its demonstration of Aletheia's autonomous capabilities in solving mathematical problems against a challenging benchmark. While the authors effectively illustrate the range of problems addressed by Aletheia, the specifics concerning the algorithms employed during problem resolution and the mechanisms of knowledge representation remain ambiguous. A critical engagement with the methodologies would have provided richer contextual depth, particularly regarding how Aletheia integrates logical reasoning and mathematical inference in problem-solving. The mention of expert assessments, although valuable, lacks robust explanatory detail regarding the structure and criteria underlying those assessments. This technical ambiguity detracts from the paper's depth, limiting the reader's ability to fully grasp the intricacies involved in Aletheia's performance.

**Research Implications**

The findings in this paper hold substantial implications for the fields of AI safety and multi-agent certification, particularly through the lens of RSCT approaches. The challenges posed by variability in expert evaluations highlight the need for standardized metrics and testing environments, which can inform future research in algorithmic accountability and robustness. Moreover, adopting RSCT methodologies could facilitate the identification and mitigation of biases induced by representation models within AI systems. Future research could enhance Aletheia or similar agents by integrating stronger methodologies that holistically assess representation quality, manage uncertainty, and rigorously evaluate noise, ultimately guiding safer deployment strategies. Furthermore, interdisciplinary collaboration between fields focusing on representation, linguistics, and epistemology can bolster the robustness of automated systems, ensuring their reliability and alignment with human expert rationale.


\subsection*{RSCT Certification Metrics}
\begin{tabular}{ll}
\textbf{Relevance (R):} & 0.374 \\
\textbf{Spurious (S):} & 0.318 \\
\textbf{Noise (N):} & 0.307 \\
\textbf{Kappa ($\kappa$):} & 0.549 \\
\end{tabular}

The kappa score of 0.549 indicates moderate representation-solver compatibility.


\section{Relevance to Swarm-It}
This paper was identified by the Swarm-It research discovery pipeline as potentially relevant to RSCT-based AI certification. The combined score of 40.8% places it in the moderate relevance category for further investigation.

\vspace{1em}
\hrule
\vspace{0.5em}
\small{Generated by Swarm-It Research Discovery | \url{https://swarms.network} | RSCT Certified}

\end{document}
